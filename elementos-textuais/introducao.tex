\chapter{Introdução}
\label{cap:introducao}

\section{Contextualização}
\label{sec:contextualizacao}
	A quantidade de dados gerados por indivíduos segundo a IBM até hoje é de 2.7 ZB (Zettabytes)\footnote{(ZB) Medida de armazenamento que corresponde a $2^{70}$ bytes. Equivale a 1.024 Exabytes,
1.048.576 Petabytes, 1.073.741.800 Terabytes ou, para ser exato, 1.180.591.620.717.411.303.424 bytes.} de dados referentes a histórico de transações, histórico de navegação, interação nas redes sociais, etc. Tendo em vista que especialista estimam que serão feitas 450 bilhões de transações pela internet todos os dia até 2020, o montante de 2.7 zb de dados armazenados será de 35 zb.

	Com o enorme volume de dados sendo gerados todos os dias, veio a necessidade de otimizar a forma como se coleta e armazena estes dados, o chamado “Big-Data”. Basicamente o “Big-Data” é uma técnica utilizada para coletar e armazenar uma grande quantidade de dados e permite alta velocidade na extração e analise de informações em uma grande variedade de tipos de dados. 

	Com a otimização do armazenamento de uma grande variedade e volume de dados, os computadores podem processar estes dados e extrair informações que possuem grande potencial na assertividade de uma tomada de decisão. Então por que não utilizar algoritmos computacionais para extrair estas informações e tomar ações proativas automaticamente? Com esta pergunta nasceu o campo de estudo “Machine Learning”, que basicamente envolve um ambiente onde exista dados disponíveis, métodos estatísticos e grande poder computacional com velocidade e paralelismo. 
	É importante ressaltar que “machine lerning” não diz respeito a encontrar informações escondidas em uma massa de dados, mas em executar tarefas da melhor maneira e de forma mais confiável, como por exemplo bloquear uma transação bancária fraudulenta ou solicitar a manutenção de uma máquina em uma fábrica antes que quebre e faça a produção parar. Mesmo que algoritmos de “machine learning” sejam muitas vezes utilizados para extrair informações de uma massa de dados, ao utilizar algoritmos de “machine learning” não se está necessariamente procurando informações. Algoritmos de “machine learning”são muito úteis em diversas áreas e influenciam muito em nosso cotidiano, como por exemplo:

	\begin{alineas}
			\item Identificar e filtrar E-mails de spam;
			\item Com base no comportamento de uso da internet pode sugerir lugares, filmes, músicas e pessoas para se fazer amizades;
			\item Identificar a rota mais rápida e segura para que as encomendas cheguem ao destino;
			\item Ensinar robôs a dirigir carros e pilotar “drones”;
			\item Automatizar os semáforos de acordo com as condições do trafego; e
			\item Automatizar o controle de trafego aéreo.
	\end{alineas}

\section{Problematização}
\label{sec:Problematizacao}

	Quais as diferenças entre os principais algoritmos de “Machine Learning”? Qual o cenário ou problema se adéquam? Por que dentre as linguagens existentes para o desenvolvimento de algoritmos de “Machine Learning”?  Quais as principais tecnologias e ferramentas existem hoje no mercado nesta área?

\section{Objetivos}
\label{sec:objetivos}
\subsection{Objetivo Geral}
\label{sec:objetivo-geral}

	O objetivo geral é analisar os principais algoritmos de “machine learning” e identificar os cenários e problemas que resolvem.


\subsection{Objetivos Específicos}
\label{sec:objetivos-especificos}

	Os objetivos específicos são: 

	\begin{alineas}
		\item Descrever o funcionamento dos principais tipos de algoritmos de “machine learning”;
		\item Apresentar quais tipos de problemas cada tipo se encaixa;
		\item Apresentar as principais tecnologias e ferramentas na área de “machine learning”; e
		\item Apresentar algumas das utilizações de “machine learning” que mais utilizamos atualmente.		
	\end{alineas}

\section{Justificativa}
\label{sec:justificativa}

Dado que existem N tipos de problemas a serem resolvidos, seja em um ambiente corporativo ou social, e uma variedade de categorias de algoritmos, pretende-se com este trabalho identificar em quais cenários e problemas os principais algoritmos de “machine learning’ são eficazes.