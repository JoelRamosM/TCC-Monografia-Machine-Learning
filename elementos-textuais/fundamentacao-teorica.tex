\chapter{Referencial Teórico}
\label{cap:referencial-reorico}

\section{O que é Machine Learning?}
\label{sec:oqueemachinelearning}

Embora a definição seja controversa, Tom M. Mitchell declara de forma plausível o que para ele é o principal objetivo de Machine Learning: "Machine Learning é o processo que faz com que um sistema melhore sua performance em determinada tarefa com base na experiência."\cite{Tom}. Podemos concluir através de sua definição que o processo de aprendizagem de um computador é semelhante ao que ocorre com nós seres humanos, onde em suma aprender é identificar padrões e reconhece-los quando vistos novamente.

Visto o objetivo de Machine Learning, para que um sistema possa identificar se sua performance está de fato melhorando é preciso identificar estes três parâmetros:
 \begin{alineascomponto}
	\item Tipo de tarefa: \textbf{\textit{T}};
	\item Métrica de performance: \textbf{\textit{P}};
	\item Treino para obter experiência: \textbf{\textit{E}};			
\end{alineascomponto}
Por exemplo, dado um computador que deve aprender a jogar xadrez os parâmetros são: \textbf{\textit{T}}: Jogar xadrez, \textbf{\textit{P}}: quantidade de jogos ganhos contra outros jogadores e
\textbf{\textit{E}}: jogar contra si.

Muitas vezes definir estes parâmetros pode ser muito complexo, por isso Machine Learning é uma área multidisciplinar, e suas principais disciplinas são: 
 
 \begin{alineascomponto}
	\item \textbf{Inteligência Artificial}, Machine Learning é um campo de AI, logo utiliza-se muitos de seus conceitos de busca e aprendizagem;  
	\item \textbf{Modelos Bayesianos}, o teorema de Bayes e o Algoritimo de Bayes são utilizados para calcular a probabilidade de hipoteses.
	\item \textbf{Teoria de Controle}, 
	\item \textbf{Teoria da Complexidade Computacional}, estuda a classificação de problemas com base na complexidade dos algoritmos que o resolvam. No ambito de ML classificasse com base no poder computacional utilizado, numero de examplos para treino, quantidade de erros, etc.   
	\item \textbf{Teoria da Informação}, 
	\item \textbf{Filosofia}, 
	\item \textbf{Psicologia}, 
	\item \textbf{Estatística}, classificar as probabilidade de hipoteses acontecerem com base em uma alguns dados, testes estatisticos.
\end{alineascomponto}