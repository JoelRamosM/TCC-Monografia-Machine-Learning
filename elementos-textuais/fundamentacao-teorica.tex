\chapter{Referencial Teórico}
\label{cap:referencial-reorico}

\section{O que é Machine Learning?}
\label{sec:oqueemachinelearning}

Embora a definição seja controversa, Tom M. Mitchell declara de forma plausível o que para ele é o principal objetivo de Machine Learning: "Machine Learning é o processo que faz com que um sistema melhore sua performance em determinada tarefa com base na experiência.". Podemos concluir através de sua definição que o processo de aprendizagem de um computador é semelhante ao que ocorre com nós seres humanos, onde em suma aprender é identificar padrões e reconhece-los quando vistos novamente.

Visto o objetivo de Machine Learning, para que um sistema possa identificar se sua performance está de fato melhorando é preciso identificar estes três parâmetros:
 \begin{alineascomponto}
	\item Tipo de tarefa: T;
	\item Métrica de performance: P;
	\item Treino para obter experiência: E;			
\end{alineascomponto}
Por exemplo, dado um computador que deve aprender a jogar xadrez os parâmetros são: T: Jogar xadrez, P: quantidade de jogos ganhos contra outros jogadores e
E: jogar contra si.

Muitas vezes definir estes parâmetros pode ser muito complexo, por isso Machine Learning é uma área multidisciplinar, e suas principais disciplinas são: 
 
 \begin{alineascomponto}
	\item Inteligência Artificial, 
	\item Estimativa Bayesiana, 
	\item Teoria de Controle, 
	\item Teoria da Complexidade Computacional, 
	\item Teoria da Informação, 
	\item Filosofia, 
	\item Psicologia, 
	\item Estatística, 
\end{alineascomponto}