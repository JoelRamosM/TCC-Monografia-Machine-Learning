\chapter{Metodologia}
\label{chap:metodologia}

Este trabalho é de carater descritivo a fim de esclarecer e descrever as caracteristicas de machine learning, 
segundo \cite{descritiva}:

\begin{citacao} 
 A pesquisa descritiva tem por objetivo descrever as características de uma população, de um fenômeno ou de uma experiência. 
 Esse tipo de pesquisa estabelece relação entre as variáveis no objeto de estudo analisado. 
 Variáveis relacionadas à classificação, medida e/ou quantidade que podem se alterar mediante o processo realizado.  
\end{citacao} 


Por meio de uma abordagem qualitativa, está pesquisa utiliza do procedimento bibliográfico, onde se utiliza de livros, 
artigos, dissertações, periódicos, internet entre outros, como fonte da pesquisa. A abordagem qualitativa segundo \cite{qualitativa} :

\begin{citacao}
A pesquisa qualitativa é traduzida por aquilo que não pode ser mensurável, pois a realidade e o sujeito são elementos indissociáveis. 
Assim sendo, quando se trata do sujeito, levam-se em consideração seus traços subjetivos e suas particularidades. 
Tais pormenores não podem ser traduzidos em números quantificáveis.
\end{citacao}