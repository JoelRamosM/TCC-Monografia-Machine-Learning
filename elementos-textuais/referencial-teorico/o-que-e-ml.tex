\section{O que é Machine Learning?}
\label{sec:oqueemachinelearning}

Embora a definição seja controversa, Tom M. Mitchell declara de forma plausível o que para ele é o principal objetivo de Machine Learning: "Machine Learning é o processo que faz com que um sistema melhore sua performance em determinada tarefa com base na experiência."\cite{Tom}. Podemos concluir através de sua definição que o processo de aprendizagem de um computador é semelhante ao que ocorre com nós seres humanos, onde em suma aprender é identificar padrões e reconhece-los quando vistos novamente.

Visto o objetivo de Machine Learning, para que um sistema possa identificar se sua performance está de fato melhorando, é preciso identificar estes três parâmetros:
 \begin{alineascomponto}
	\item Tipo de tarefa: \textbf{\textit{T}};
	\item Métrica de performance: \textbf{\textit{P}};
	\item Treino para obter experiência: \textbf{\textit{E}};			
\end{alineascomponto}
Por exemplo, dado um computador que deve aprender a jogar xadrez os parâmetros são: \textbf{\textit{T}}: Jogar xadrez, \textbf{\textit{P}}: quantidade de jogos ganhos contra outros jogadores e
\textbf{\textit{E}}: jogar contra si.

Muitas vezes definir estes parâmetros pode ser muito complexo, mesmo porque ainda não conseguimos compreender 100\% os processos
de aprendizagem em seres humanos, por isso os processos de aprendizagem de máquina engloba várias disciplinas, e suas principais disciplinas são: 
 
 \begin{alineascomponto}
	\item \textbf{Inteligência Artificial}, Machine Learning é uma subárea de IA\footnote{IA é um acrônimo para Inteligência Artificial,
	 que é um ramo de pesquisa da ciência da computação que busca, através de símbolos computacionais, construir mecanismos 
	 e/ou dispositivos que simulem a capacidade do ser humano de pensar, resolver problemas, ou seja, de ser inteligente.\cite{IA_Web}},
	 logo utiliza-se muitos de seus conceitos de busca e aprendizagem;  
	\item \textbf{Modelos Bayesianos}, o teorema de Bayes e o Algoritimo de Bayes são utilizados para calcular a probabilidade de hipóteses.
	\item \textbf{Estatística}, classificar as probabilidade de hipóteses acontecerem com base em dados, testes estatisticos, classificação, etc.
	\item \textbf{Teoria da Complexidade Computacional}, estuda a classificação de problemas com base na complexidade dos algoritmos que
	o resolvam. Na área de machine learning classificasse com base no poder computacional utilizado, numero de examplos para treino, quantidade de erros,etc.   
	\item \textbf{Teoria de Controle}, trata do problema de estabelecer critérios de estabilidade, assim como desenvolver estratégias 
	relevantes de controle e filtragem , para sistemas dinâmicos sujeitos a incertezas.
	\item \textbf{Filosofia da Ciência}, existe um relação muito clara entre a escolha de uma hipótese e a escolha de um modelo em machine learning. Visto que Filosofia da ciência é a área da filosofia que pergunta sobre a ciência, de quais ideias parte, qual método usa, sobre qual fundamento e acerca de suas implicações \cite{FilosofiaCiencia}.
	\item \textbf{Teoria da Informação},A teoria da informação é um ramo da matemática que estuda quantificação da informação.Essa teoria teve seus 
	pilares estabelecidos por Claude Shannon(1948) que formalizou conceitos com aplicações na teoria da comunicação e estatística. 
	A teoria da informação foi desenvolvida originalmente para compressão de dados, para transmissão e armazenamento destes. 
	Porém, foi planejada para aplicação ampla, e têm sido usada em muitas outras áreas.\cite{TeoriaInformacao}	
\end{alineascomponto}

A àrea de Aprendizagem de máquina está interessada no desenvolvimento de algoritmos que transformam dados em ações inteligentes, oque causou
um ciclo de avanços nas seguintes áreas: armazenamento de dados, métodos estatisticos e poder computacional. Está relação se da pelo aumento de dados disponíveis que necessita de 
mais poder computacional, que por sua vez permite o desenvolvimento de métodos estatísticos para analizar grandes massas de dados. Um dos principais motivos destes avanços serem fomentados, 
é porque as grandes corporações acreditam que com a utilização de ferramentas e automações irão errar menos em suas tomadas de decisão, e com isso poupar dinheiro e aumentar seus lucros.

