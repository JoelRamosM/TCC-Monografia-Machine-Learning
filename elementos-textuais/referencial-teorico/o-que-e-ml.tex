\section{O que é Machine Learning?}
\label{sec:oqueemachinelearning}

Embora a definição seja controversa, Tom M. Mitchell declara de forma plausível o que para ele é o principal objetivo de Machine Learning: Machine Learning é o processo que faz com que um sistema melhore
sua performance em determinada tarefa com base na experiência\cite{Tom}. Pode-se concluir através de sua definição que o processo de aprendizagem de um 
computador é semelhante ao que ocorre com nós seres humanos, onde em suma aprender é identificar padrões e reconhece-los quando vistos novamente.

Visto o objetivo de Machine Learning, para que um sistema possa identificar se sua performance está de fato melhorando, é preciso identificar estes três parâmetros:
 \begin{alineascomponto}
	\item Tipo de tarefa: \textbf{\textit{T}};
	\item Métrica de performance: \textbf{\textit{P}};e
	\item Treino para obter experiência: \textbf{\textit{E}}.			
\end{alineascomponto}
Por exemplo, dado um computador que deve aprender a jogar xadrez os parâmetros são: \textbf{\textit{T}}: Jogar xadrez, \textbf{\textit{P}}: quantidade de jogos ganhos contra outros jogadores e
\textbf{\textit{E}}: jogar contra si.

Muitas vezes definir estes parâmetros pode ser muito complexo, mesmo porque ainda não conseguiu-se compreender 100\% os processos
de aprendizagem em seres humanos, por isso os processos de aprendizagem de máquina engloba várias disciplinas, e suas principais disciplinas são: 
 
 \begin{alineascomponto}
	\item \textbf{Inteligência Artificial}: Machine Learning é a principal subárea de IA\footnote{IA é um acrônimo para Inteligência Artificial,
	 que é um ramo de pesquisa da ciência da computação que busca, através de símbolos computacionais, construir mecanismos 
	 e/ou dispositivos que simulem a capacidade do ser humano de pensar, resolver problemas, ou seja, de ser inteligente.\cite{IA_Web}},
	 logo utiliza-se muitos de seus conceitos;  
	\item \textbf{Modelos Bayesianos}: o teorema de Bayes e o Algoritimo de Bayes são utilizados para calcular a probabilidade de hipóteses;
	\item \textbf{Estatística}: classificar as probabilidade de hipóteses acontecerem com base em dados, testes estatisticos, classificação, etc.;
	\item \textbf{Teoria da Complexidade Computacional}: estuda a classificação de problemas com base na complexidade dos algoritmos que
	o resolvam. Na área de machine learning classificasse com base no poder computacional utilizado, numero de examplos para treino, quantidade de erros, etc.;   
	\item \textbf{Teoria de Controle}: trata do problema de estabelecer critérios de estabilidade, assim como desenvolver estratégias 
	relevantes de controle e filtragem, para sistemas dinâmicos sujeitos a incertezas;
	\item \textbf{Filosofia da Ciência}: existe um relação muito clara entre a escolha de uma hipótese e a escolha de um modelo em machine learning. Visto que Filosofia da ciência é a área da filosofia que pergunta sobre a ciência, de quais ideias parte, qual método usa, sobre qual fundamento e acerca de suas implicações \cite{FilosofiaCiencia}; e
	\item \textbf{Teoria da Informação}: a teoria da informação é um ramo da matemática que estuda quantificação da informação.Essa teoria teve seus 
	pilares estabelecidos por Claude Shannon(1948) que formalizou conceitos com aplicações na teoria da comunicação e estatística. 
	A teoria da informação foi desenvolvida originalmente para compressão de dados, para transmissão e armazenamento destes. 
	Porém, foi planejada para aplicação ampla, e têm sido usada em muitas outras áreas.\cite{TeoriaInformacao}.	
\end{alineascomponto}

A área de aprendizagem de máquina está interessada no desenvolvimento de algoritmos que transformam dados em ações inteligentes, oque causou
um ciclo de avanços nas seguintes áreas: armazenamento de dados, métodos estatisticos e poder computacional. Esta relação se da pelo aumento de dados disponíveis que necessita de 
mais poder computacional, que por sua vez permite o desenvolvimento de métodos estatísticos para analizar grandes massas de dados. Um dos principais motivos destes avanços serem fomentados, 
é a visão das grandes corporações sobre a utilização de ferramentas e automações, que farão com que errem menos em suas tomadas de decisão, e com isso poupar dinheiro e aumentar seus lucros.

\subsection{Exemplos de uso de Machine Learning}
\label{cap:exemplos-ml}

Para se ter uma visão mais ampla sobre as possibilidade de aplicações de machine learning,
 é interessante conhecer algumas de suas aplicações que estão presentes no nosso dia-a-dia e outras que tem o propósito de 
 mudar o modo como vivemos, são elas:

 \begin{alineas}
	\item \textbf{Reconhecimento de caracteres em imagens}, um exemplo muito utilizado atualmente por instituições de segurança e de trânsito, para 
	reconhecimento de placas de veículos através de câmeras nas ruas,estradas e rodovias.  
	\item \textbf{Reconhecimeto de formas/objetos (Visão Computacional)\footnote{\textbf{Visão computacional} é a ciência e tecnologia das máquinas que enxergam. Ela desenvolve teoria e tecnologia para a construção de sistemas artificiais que obtém informação de imagens ou quaisquer dados multi-dimensionais.\cite{vscomp}}}, está aplicação pode ser útil em vários aspectos, porém acredito que a mais importante 
	é a melhoria na qualidade de vida de pessoas com deficiência visual, onde basicamente a pessoa usa um óculos com uma câmera a qual faz o reconhecimento do ambiente
	e o descreve, em linguagem natural.  
	\item \textbf{Reconhecimento de fala}, mais uma das aplicações de machine leraning de grande valia para a melhoria na qualidade de vida das pessoas,
	levando em consideração o seu papel na inclusão tecnologica/social de pessoas com alguma deficiência motora que á impessa de interagir 
	com interfaces por meios convencionais, como o teclado ou  mouse do computador, e dispositivos móveis com interface \textit{touchscreen}.
	\item \textbf{Reconhecimento facial}, muito utilizada em orgão de segurança para identificar criminosos procurados ou pessoas desaparecidas.
	\item \textbf{Diagnóstico médico}, aumenta a probabilidade do médico acertar o diagnóstico do paciente dado seus sintomas.   
	\item \textbf{Previsão do clima}, basicamente determinar se amanhã choverá ou não, com base em dados de anos anteriores, ou até mesmo como será a 
	safra de soja nos próximos anos. 
	\item \textbf{Detecção de fraude de transações financeiras}, determina com base em métodos estatísticos, a probabilidade de uma transação ser fraudulenta.
 \end{alineas}

 \subsection{Conceitos importantes}
 \label{subsec:conceitos}
 Antes de começar a entender como funciona o processo de aprendizagem de máquina deve-se estar a par de alguns conceitos importantes 
 utilizados no meio, são eles:
  \begin{alineas}
  	\item \textbf{Exemplo/Instância}, também conhecido como \textbf{dado de treino}, é o dado á ser classificado e a saída da etapa de pré=processamento dos dados.
	Por exemplo em um sistema de visão computacional as imagens são os exemplos.
	\item \textbf{Atributo}, é uma propriedade de um exemplo que pode ser usada para classifica-lo, também chamado de \textit{feature}. Seguindo o mesmo exemplo do item anterior,
	a forma, tamanho ou cor são atribudos de um exemplo de imagem. Determinar quais atributos são valiosos para uma determinada tarefa pode 
	ser difícil.
	\item \textbf{Modelo}, é uma representação explícita e resumida dos padrões encontrados entre os dados. A maioria dos algoritmos de indução gerão modelos,
	para serem usados como classificadores, regressores ou elementos para concepção humana.
	\item \textbf{Algoritmo de indução/indutor}, são algoritmos que utilizam exemplos e produzem modelos que generalizam estes exemplos.
	\item \textbf{Classificador}, è a saída de um algoritmo de aprendizagem/indutor, que pode ser usado para classificar.	
	\item \textbf{\textit{Bias}/Tendência}, pode-se dizer que é um peso dado aos modelos para ajusta-lo, para que se possa induzir o modelo escolhido
	 a uma determinada expectativa. Em suma é qualquer preferência de uma hipótese sobre outra.
	 \item \textbf{\textit{Overfitting}}, é uma situação a qual o modelo sofre vários ajustes, oque faz que ele seja muito preciso, 
	 porém é preciso evitar modelos  esta situação pois estes modelos não refletem perfeitamente a realidade.
  \end{alineas}  
