\section{O Processo para aplicar machine learning}
\label{sec:howtoapplyml}

Antes de começar o processo é preciso estar atento à alguns pressupostos já consolidado na comunidade de machine learning, pois estes lhe darão insumos muito importantes durante 
as tomadas de decisões do projeto, são eles: 
\begin{alineas}
    \item O processo é \textbf{Iterativo}, lembre do processo de generalização onde temos de treinar os modelos afim de identificar os erros nos modelos e corrigi-los para que não cause
    anomalias nos resultados, este processo é feito várias vezes até termos um resultado aceitável;
    \item É  \textbf{desafiador}, raramente será fácil. Por exemplo identificar o conjunto de dados que será treinado, é preciso de um cientista de dados para isso; e 
    \item Nem sempre terá o resultado esperado, porém quando se tem sucesso é um ganho substancial.
\end{alineas}
  
A etapa mais importante do processo é identificar a pergunta certa para o problema que deseja resolver, e a razão de ser a etapa mais importante é 
porque se a pergunta certa não for feita não terá a resposta esperada, por exemplo um sistema que identifica transações de cartão de crédito fraudulentas, a 
pergunta basicamente é "\textit{Está é uma transação fraudulenta?}". Após isso deve questionar se possue os dados corretos para responder a pergunta e como determinar o sucesso dos resultados,
é preciso muito cuidado em como medir o sucesso, pois se a métrica não for identificada corretamente nunca saberá se o algoritmo está funcionando.


