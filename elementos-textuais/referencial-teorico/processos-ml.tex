\section{O Processo para aplicar machine learning}
\label{sec:howtoapplyml}

É importante ter em mente as caracteristicas do processo de machine learning é:
\begin{alineas}
    \item \textbf{Iterativo}, lembre do processo de generalização onde temos de treinar os dados afim de identificar os erros nos modelos e corrigi-los para que não cause
    anomalias nos resultas, este processo é feito várias vezes até termos um resultado aceitável;
    \item É  \textbf{desafiador}, raramente será fácil. Por exemplo quais parâmetros utilizar para identificar 
    o conjusto de dados que será treinada; 
    \item Nem sempre terá o resultado esperado, porém quando se tem sucesso é um ganho substancial;
\end{alineas}
  
A etapa mais importante do processo é identificar a pergunta certa para o problema que deseja resolver, por exemplo um sistema que identifica transações de cartão de crédito fraudulentas, a 
pergunta basicamente é "\textit{Está é uma transação fraudulenta?}". Após isso deve questionar se possue os dados corretos para responder a pergunta e como determinar o sucesso do resultado,
é preciso muito cuidado em como medir o sucesso, pois se não for identificado corretamente nunca saberá se o algoritmo funciona. 