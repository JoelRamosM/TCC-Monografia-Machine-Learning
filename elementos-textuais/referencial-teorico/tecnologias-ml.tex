\section{Tecnologias para Machine Learning}
\label{sec:tech-ml}

Machine Learning é uma área que vem sendo explorada e aprimorada a cada dia por cientistas tanto da iniciativa privada 
como pública e acadêmica. Com a rápida evolução a cada dia são desenvolvidas novas tecnologias para o desenvolvimento
de métodos e algoritmos de machine learning, para ilustrar melhor é possivel  dividir estas técnologias em categorias, são elas
\textbf{linguagens de programação} e \textbf{\textit{frameworks}}, \textbf{ferramentas} e \textbf{serviços}. 

\subsection{Principais linguagens de programação e frameworks}
\label{subsec:ling-prog}

É possível desenvolver um algoritmo utilizando qualquer linguagem de programação, mas é de suma importância que entenda as implicações
em relação á escalabilidade, performance, sintaxe e paradigmas atendidos pela linguagem escolhida. 

Muitos desenvolvedores optam por desenvolver um algoritmo com a linguagem a qual possuem mais familiaridade, 
porém dependendo da linguagem pode haver algumas caracteristicas que  oneram a performance, como por exemplo linguagens que possuem 
\textit{Garbage Collectors} \footnote{\textbf{\textit{Garbage collector}} é um mecanismo que verifica espaços alocados na memoria que não estão mais sendo utilziados e os remove.} como Java, C\#, Python entre outras. É aconselhavel que o desenvolvedor válide a 
idéia com protótipos desenvolvidos com a linguagem que possua mais familiaridade, isto torna o processo mais rápido, porém e necessário
validar se será necessário refazer o código para produção em uma outra linguagem mais performática, pois em alguns casos não à necessidade
de reescrever o código em outra linguagem pois o protótipo atende os critérios desejados de performance e escalabilidade. 

A linguagem de programação não influenciará no resultado final do algoritmo, porém influencia no tempo de desenvolvimento e 
velociade de processamento, alguns critérios comuns são adotados na escolha:

\begin{alineas}
    \item Sintaxe, pois influencia diretamente na manuntenção e entendimento do código, uma boa sintaxe torna mais fácil a evolução e melhora do algoritmo.
    \item Funcionalidades dentro da linguagem, como polimorfismo, estruturas de dados como \textbf{\textit{hash tables}}, etc. 
    \item Bibliotecas e Frameworks disponíveis, é interessante que exista bibliotecas de estatistica, algebra, graficas, \textit{I/O} e grafos. Para que
          o desenvolvedor não se preocupe em reinventar a roda e utilizar funções de cálculos ou leituras de uma imgem prontas, e foque no algoritmo de 
          aprendizado.
    \item Familiaridade, não somente a do desenvolvedor mas de toda pessoa que possa ver o código.     
\end{alineas}

Foi realzada uma pesquisa em 2015 pela \textbf{KDNuggets} \footnote{\textbf{KDNuggets} é um site muito famoso nas áreas de Big Data, Data Mining e Data Science.}, para identificar qual a linguagem mais popular dentre os desenvolvedores de algortimos para 
machine learning, a Figura ~\ref{fig:poll-lang} mostra o resultado da pesquisa, constatou-se que a linguagem \textbf{R} é a mais utilizada seguida de 
\textbf{Python}.   


A linguagem \textbf{R} é mais utilizada para estes tipos algortimos porque foi projetada para computação estatistica, portanto 
nativamente possui muitas funções uteis para cálculos estatisticos e processamento de muitos dados, é 
altamente escalável oque é otimo para mineração de dado em uma grande massa de dados, além de possuir um grande repositório com 
bibliotecas que facilitam a aplicação da maioria dos tipos de algoritmos para machine learning e testes estatisticos o 
\textbf{CRAN}(\textit{Comprehensive R Archive Network}). 
Possui uma sintaxe elegante para transfomar dados, expressar relacionamentos e é muito fácil criar operações paralelas.
\textbf{CRAN} é um repositório de bibliotecas para a linguagem \textbf{R}, atualmente(2016) possui 9481 bibliotecas ativas, 
onde parte são focadas em machine learning.


Porém \textbf{Python} é uma linguagem mais generalista, atendendo diversos tipos de desenvolvimento como web, descktop e serviços, 
é tambem uma linguagem de script ou seja e interpredada em tempo de execução e não compilada. 
Isto fez com que ficasse muito popular entre os cientistas de dados e engenheiros de machine learning, mesmo que diferente de 
\textbf{R} não possua nativamente funções estatisticas existem bibliotecas que oferecem estas mesmas funções com uma sintaxe
indiscutivelmente mais simples, no repositório \textbf{PyPI}(\textit{Python Package Index}) que atualmente possui 92380 bibliotecas disponíveis.      

\begin{figure}[h!]
	\centering
	\Caption{\label{fig:poll-lang} Linguagem mais utilizada em ML }	
	\UECEfig{}{
		\fbox{\includegraphics[width=15cm]{figuras/poll-lang}}
	}{
		\Fonte{KDNuggets 2015 poll: Primary programming language for Analytics, Data Mining, Data Science tasks}
	}	
\end{figure}

Além das linguagens e suas bibliotecas citadas, exitem alguns frameworks muito usados pela comunidade de engenheiros de machine learning, 
que não necessáriamente utilizam as linguagens mais comuns, os principais são \textbf{Apache Spark} e o \textbf{Apache Hadoop.}

O framework \textbf{Apache Spark} é voltado para processamento de \textbf{Big Data}, feito para ser usual, rápido e para o uso de análises mais
sofisticadas. É um projeto da \textbf{Fundação Apache}, que é uma organização sem fins lucrativos afim de suportar 
projetos com código livre, logo o \textbf{Spark} é um projeto de código livre. Este vem sendo desenvolvido desde 2009 pelo 
AMPLab da Universidade de Califórnia em Berkeley e em 2010 seu código foi aberto como projeto da fundação Apache. 
O Spark é um framework consolidado que oferece formas de gerenciar e processar Big Data com varias naturezas de dados, como textos ou imagens, e várias origens,  
como lote ou streaming de dados,  de forma muito compreeendivel. Além de melhorar a performance de aplicações e dar mais produtividade no desenvolvimento
de aplicações em Java, Scala, R e Python. Posto seu foco no alto desempenho Spark possui uma bibliteca chamada \textbf{MLlib} para Machine Learning a qual 
consiste nos algoritmos de Machine Learning, como classificação, regressão e clustering.

Assim como o Spark o \textbf{Apache Hadoop} nasceu com a necessidade de processar uma grande quantidade de Big Data e tambem é um \textit{software} 
livre com o código aberto, porém originou-se do framework \textbf{MapReduce} da Google  criado para indexar páginas web, porém foi muito mais além 
, e é utilizado por industrias das áreas de entreterimento, mercado financeiro, governamentais, saúde, informação e outras. Basicamente ele utilza
técnicas do MapReduce para consultar dados e dividir e processa-los paralelamente em vários computadores, oque permite processar muito mais dados em menos
tempo. Assim como o Hadoop o Spark também estende funcionalidades do MapReduce.
 

\subsection{Principais ferramentas}
\label{subsec:ferramentas}
Muitas vezes não é necessáro desenvolver um sitema de machine learning para que se obtenha resultados satisfátórios, pois existem
ferramentas focadas em automatizar algumas tarefas para fazer \textbf{data-mining}, preparar os dados ou visualiza-los de formas mais 
simples, além de ferramentas que facilitam o desenvolvimento de algortimos. 


  






    