Machine Learning é considerada a principal sub-área de IA (Inteligêcia Articial), que tem como principal objetivo fazer com 
que um computador aprenda sozinho como realizar dada tarefa, sem que os passos desta tarefa sejam previamente programados e
quando receber novos dados possa alterar a forma como realiza esta tarefa a fim de otimizar o processo. Possui uma divisão
em categorias, que é dada pela organização dos dados utilziados para a aprendizagem, as categorias são aprendizagem supervisionada, não-supervisionada, 
semi-supervisionada e por reforço.   


Está área fomenta o avanço e desenvolvimento de novos métodos, algoritmos e tecnologias. Tanto o meio acadêmico quanto o meio 
corporativo entendem as possibilidades ao aplicar machine learning, no entanto o meio corporativo é o principal motivador 
dos avanços, pois é possivel que ao aplicar machine learning em dados de um mercado especifico é possível obter uma grande
vantagem competitiva. Além de grandes empresas que oferecem serviços para abstrair a complexidade de um projeto de 
machine learning e facilitar sua aplicação. 

Trata-se de uma pesquisa de cunho qualitativo utilizando do procedimento
bibliográfico tendo como referência artigos, livros publicações na internet sobre o assunto. Pode-se concluir a partir da 
leitura deste trabalho que existem vários métodos de aplicações de \textit{machine learning} e a escolha do que mais 
se adequa ao contexto é muito importante, pois depende da organização dos dados utilizados. 
Além de apresentar as principais ferramentas e tecnologias da área.   



% Separe as palavras-chave por ponto
\palavraschave{IA;Machine Learning;}