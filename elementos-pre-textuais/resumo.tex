Machine Learning é considerada a principal sub-área de IA (Inteligêcia Articial), pois permite
que um computador aprenda sozinho como realizar dada tarefa, sem que os passos desta tarefa sejam previamente programados, e
quando recebem novos dados podem alterar a forma como realizam esta tarefa a fim de melhorar o processo, ou seja possui a 
habilidade de se adaptar a novos dados. Está presente no dia-a-dia da maioria das pessoas desde o trajeto traçado no GPS
para ir ao trabalho até sugestões de filmes do seu gosto.  


Isto colabora para que a cada dia novos métodos e algoritmos sejam desenvolvidos afim de responder perguntas impossíveis de se responder
analisando a crescente massa de dados disponíveis manualmente. Além das grandes empresas acreditarem que exista muito dinheiro nesta área,
por isso investem fortemente, para que tenham vantagens de mercado, seja uma empresa que deseja ter informações sobre
seus clientes afim de direcionar campanhas de marketing ou de empresas de serviços que oferecem uma forma fácil de se implementar
uma tarefa de machine learning.

Machine learning possui diversos métodos de como ser aplicado dependendo dos tipos de dados e a pergunta que deseja responder, 
é importante estar atento para estes métodos, pois aplicando métodos diferentes nos mesmos dados o resultado será diferente.

% Separe as palavras-chave por ponto
\palavraschave{IA;Machine Learning;}